\documentclass{article}
\usepackage[T1]{fontenc}
\usepackage[utf8]{inputenc}

\usepackage{amsmath} 
\usepackage{amsfonts}
\usepackage{amssymb}

\usepackage{graphicx}
\usepackage{tikz}

\usepackage[ngerman]{babel}

\setlength{\parindent}{0pt}

\begin{document}
\section{Satz von Pythagoras}

\begin{figure}[h]
\centering
\begin{tikzpicture}
\draw[help lines] (0,0) grid (8,8);
\draw [semithick, draw=black, fill=red, fill opacity=0.2]
       (0,3) -- (5,0) -- (8,5) -- (3,8) -- cycle;
\draw [semithick, draw=black, fill=yellow, fill opacity=0.2]
       (0,0) -- (0,3) -- (5,0) -- cycle;
\draw [semithick, draw=black, fill=green, fill opacity=0.2]
       (5,0) -- (8,0) -- (8,5) -- cycle;
\draw [semithick, draw=black, fill=green, fill opacity=0.2]
       (3,8) -- (8,8) -- (8,5) -- cycle;
\draw [semithick, draw=black, fill=green, fill opacity=0.2]
       (0,8) -- (3,8) -- (0,3) -- cycle;
\node at (4,4) {$c^2$};
\node[left] at (0,1.5) {$a$};
\node[below] at (2.5,0) {$b$};
\node[above,right] at (2.5,1.7) {$c$};
\node[below] at (6.5,0) {$a$};

\end{tikzpicture}
\end{figure}

Seien $a, b, c$ die Seiten eines Dreiecks mit der Seite $c$ (Hypotenuse), die sich stets gegenüber einem $90^\circ$-Winkel befindet, den $b$ auf $a$ bildet, dann ist das Quadrat über $c$ flächengleich zu der Summe der Quadrate über $a$ und $b$.

\begin{equation}
a^2 + b^2 = c^2
\end{equation}

\section{Kathetensatz des Euklid}

Seien $a, b, c$ die Seiten eines rechtwinkligen Dreiecks mit der Hypotenuse $c$. Teilt man dieses Dreieck an der Höhe $h$ und mit $p$ der Hypotenusenabschnitt über $a$, $q$ der entsprechende Abschnitt über $b$, so gilt:
Das Quadrat über a ist flächeninhaltsgleich zum Rechteck mit den Seiten $p$ und $c$, und das Quadrat über $b$ ist flächeninhaltsgleich zum Rechteck mit den Seiten $q$ und $c$.

\begin{align}
a^2 &= p\cdot c\\
b^2 &= q\cdot c
\end{align}

\section{H"ohensatz des Euklid}

Gegeben sei ein rechtwinkliges Dreieck mit der Höhe h, welche die Hypotenuse in die Abschnitte p und q teilt. Dann ist:

\begin{equation}
h^2 = p\cdot q
\end{equation}

\end{document}