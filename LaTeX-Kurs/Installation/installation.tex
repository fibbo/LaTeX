\documentclass[10pt, xcolor={table}]{beamer}
\usepackage[]{graphicx}
\usepackage[]{color}
%% maxwidth is the original width if it is less than linewidth
%% otherwise use linewidth (to make sure the graphics do not exceed the margin)
\makeatletter
\def\maxwidth{ %
  \ifdim\Gin@nat@width>\linewidth
    \linewidth
  \else
    \Gin@nat@width
  \fi
}
\makeatother

\usepackage{listings}


\newcommand{\backupbegin}{
   \newcounter{framenumberappendix}
   \setcounter{framenumberappendix}{\value{framenumber}}
}
\newcommand{\backupend}{
   \addtocounter{framenumberappendix}{-\value{framenumber}}
   \addtocounter{framenumber}{\value{framenumberappendix}} 
}



\usepackage{alltt}

\usepackage{url,verbatim}
\usepackage[ngerman]{babel}

\usepackage{avant}
\renewcommand*\familydefault{\sfdefault} %% Only if the base font of the document is to be sans serif
\usepackage[T1]{fontenc}

\usepackage{tikz}
\usetikzlibrary{positioning,shapes,arrows,decorations.pathreplacing,calc,automata}
% load extra stuff
% \usetikzlibrary{shapes.misc,matrix,positioning,fit}

% cross out text 
\usepackage[normalem]{ulem}


\usepackage[headheight=22pt]{beamerthemeboxes}
%\usepackage{bbm}
\usepackage{graphicx}
\beamertemplatenavigationsymbolsempty 
\setbeamercovered{transparent}

% \definecolor{redve}{rgb}{0.604,0.008,0.00}
\definecolor{redve}{rgb}{0,0.61,0.245}
% \definecolor{uzh}{rgb}{0.188,0.522,0.306}
\definecolor{uzh}{HTML}{0028a5}
\setbeamertemplate{itemize item}{$\bullet$} 
\setbeamercolor{title}{fg=uzh}
\setbeamercolor{frametitle}{fg=uzh}
\setbeamertemplate{sections/subsections in toc}[ball unnumbered]
\setbeamercolor{section in toc}{fg=uzh,bg=white}
\setbeamercolor{subsection in toc}{fg=uzh,bg=white}
\setbeamercolor{result}{fg=black, bg=yellow}



\addheadboxtemplate{\color[rgb]{1,1,1}}{\color{uzh} \underline{{\hspace{3pt}\includegraphics[scale=0.33]{uzh_logo_e_pos} 
\hspace{0.55\paperwidth}\color{black} \tiny Einf"uhrung in \LaTeX} \hspace{11pt}}}

\addfootboxtemplate{\color[rgb]{1,1,1}}{\color{black} %\tiny \quad  %bla 
\hfill \tiny
\insertframenumber / \inserttotalframenumber \hspace{5pt}}


%% set item distance
\newlength{\wideitemsep}
\setlength{\wideitemsep}{\itemsep}
\addtolength{\wideitemsep}{0.3em}
\let\olditem\item
\renewcommand{\item}{\setlength{\itemsep}{\wideitemsep}\olditem}



% \titlegraphic{\includegraphics[width=0.3\textwidth]{uzh_logo_e_pos}\hspace*{4.75cm}}


\usepackage{amsmath}

\title[Installation von \LaTeX] % (optional, only for long titles)
{Installation von \LaTeX}
\author{Philipp Gloor}
\institute[] % (optional)
{\inst{1}philipp.gloor@gmail.com}
\date[Feburar 2016] % (optional)
{\LaTeX{} Einf"uhrung}

\begin{document}

\begin{frame}
  \titlepage
\end{frame}
\begin{frame}{Was braucht es f"ur \LaTeX}
\LaTeX ist nicht ein herk"ommliches Programm wie z.B. Word von Microsoft. Das heisst, es gibt keine
\texttt{latex.exe} Datei, die man ausf"uhrt und dann mit \LaTeX arbeitet.
\begin{itemize}
  \item \LaTeX-Distribution (z.B. texlive oder Miktex)
  \item (optional) Editor
\end{itemize}

Die Distribution ist das Herzst"uck von \LaTeX. Ohne die Distribution geht nichts. Der Editor
ist lediglich ein Hilfsmittel, aber ein sehr wichtiges.
\end{frame}


\section{Distributionen}
\begin{frame}{Distributionen}
Wir werden die Distribution von texlive verwenden. Sie ist verf"ugbar f"ur alle g"angigen 
Betriebssysteme (Windows, Mac, Linux).

\begin{itemize}
  \item texlive f"ur Windows: \url{https://www.tug.org/texlive/acquire-netinstall.html}
  \item texlive f"ur Mac: \url{https://tug.org/mactex/mactex-download.html}
\end{itemize}
\end{frame}

\begin{frame}{Editor}
  Ein Editor ist nicht zwingend notwendig, erleichtert das Arbeiten mit \LaTeX aber immens. Es gibt eine vielzahl
  von verschiedenen Editoren (\url{https://en.wikipedia.org/wiki/Comparison_of_TeX_editors}).

  Einer der besten:
  \begin{itemize}
    \item texmaker (\url{http://www.xm1math.net/texmaker/})
  \end{itemize}
\end{frame}

\begin{frame}{texmaker}
  
\end{frame}

\end{document}
