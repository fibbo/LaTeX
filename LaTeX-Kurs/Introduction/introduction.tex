\documentclass[10pt, xcolor={table}]{beamer}
\usepackage[T1]{fontenc}
\usepackage[utf8]{inputenc}
\usepackage[]{graphicx}
\usepackage{url,verbatim}

\usepackage[]{xcolor}
\usepackage{hyperref}
\hypersetup{urlcolor=-green!50}
%% maxwidth is the original width if it is less than linewidth
%% otherwise use linewidth (to make sure the graphics do not exceed the margin)
\makeatletter
\def\maxwidth{ %
  \ifdim\Gin@nat@width>\linewidth
    \linewidth
  \else
    \Gin@nat@width
  \fi
}
\makeatother

\usepackage{listings}


\newcommand{\backupbegin}{
   \newcounter{framenumberappendix}
   \setcounter{framenumberappendix}{\value{framenumber}}
}
\newcommand{\backupend}{
   \addtocounter{framenumberappendix}{-\value{framenumber}}
   \addtocounter{framenumber}{\value{framenumberappendix}} 
}



\usepackage{alltt}
\usepackage{parskip}
\usepackage[ngerman]{babel}

\usepackage{avant}
\renewcommand*\familydefault{\sfdefault} %% Only if the base font of the document is to be sans serif
\usepackage[T1]{fontenc}

\usepackage{tikz}
\usetikzlibrary{positioning,shapes,arrows,decorations.pathreplacing,calc,automata}
% load extra stuff
% \usetikzlibrary{shapes.misc,matrix,positioning,fit}

% cross out text 
\usepackage[normalem]{ulem}

\usepackage{listings}
\definecolor{mygreen}{rgb}{0,0.6,0}
\definecolor{mygray}{rgb}{0.9,0.9,0.9}
\definecolor{mymauve}{rgb}{0.58,0,0.82}
\lstset{ %
  backgroundcolor=\color{mygray},   % choose the background color; you must add \usepackage{color} or \usepackage{xcolor}
  basicstyle=\footnotesize\ttfamily,        % the size of the fonts that are used for the code
  breakatwhitespace=false,         % sets if automatic breaks should only happen at whitespace
  belowskip=-2em,
  breaklines=true,                 % sets automatic line breaking
  captionpos=b,                    % sets the caption-position to bottom
  commentstyle=\color{mygreen},    % comment style
  deletekeywords={...},            % if you want to delete keywords from the given language
  escapeinside={\%*}{*)},          % if you want to add LaTeX within your code
  extendedchars=true,              % lets you use non-ASCII characters; for 8-bits encodings only, does not work with UTF-8
  frame=single,                    % adds a frame around the code
  keepspaces=true,                 % keeps spaces in text, useful for keeping indentation of code (possibly needs columns=flexible)
  keywordstyle=\color{red},       % keyword style
  language=tex,                 % the language of the code
  morekeywords={begin, end, documentclass,usepackage,author, maketitle, newpage, title, date, linespread, selectfont, section, subsection, subsubsection, paragraph, subparagraph, tableofcontents, item, verb, setlength, parindent, parskip, cite, parencite, footcite, makeindex, printindex, includegraphics, textwidth, textbf, cline, multirow, multicolumn, hline, footnote,blindtext,renewcommand, pagestyle, label, ref, nameref, pageref, makebox, caption, fbox, framebox, mbox, raggedleft, raggedright, centering, textrm, textsl, textit, textsc, textsf, texttt, setcounter},            % if you want to add more keywords to the set
  numbers=left,                    % where to put the line-numbers; possible values are (none, left, right)
  numbersep=5pt,                   % how far the line-numbers are from the code
  numberstyle=\tiny\color{mygray}, % the style that is used for the line-numbers
  rulecolor=\color{black},         % if not set, the frame-color may be changed on line-breaks within not-black text (e.g. comments (green here))
  showspaces=false,                % show spaces everywhere adding particular underscores; it overrides 'showstringspaces'
  showstringspaces=false,          % underline spaces within strings only
  showtabs=false,                  % show tabs within strings adding particular underscores
  stepnumber=2,                    % the step between two line-numbers. If it's 1, each line will be numbered
  stringstyle=\color{mymauve},     % string literal style
  tabsize=2,                       % sets default tabsize to 2 spaces
  title=\lstname                   % show the filename of files included with \lstinputlisting; also try caption instead of title
}         % Palatino needs more leading (space between lines)

\lstset{
  literate={ö}{{\"o}}1
           {ä}{{\"a}}1
           {ü}{{\"u}}1
           {Ü}{{\"U}}1
}

\usepackage[headheight=22pt]{beamerthemeboxes}
%\usepackage{bbm}
\usepackage{graphicx}
\beamertemplatenavigationsymbolsempty 
\setbeamercovered{transparent}

% \definecolor{redve}{rgb}{0.604,0.008,0.00}
\definecolor{redve}{rgb}{0,0.61,0.245}
% \definecolor{uzh}{rgb}{0.188,0.522,0.306}
\definecolor{uzh}{HTML}{0028a5}
\setbeamertemplate{itemize item}{$\bullet$} 
\setbeamercolor{title}{fg=uzh}
\setbeamercolor{frametitle}{fg=uzh}
\setbeamertemplate{sections/subsections in toc}[ball unnumbered]
\setbeamercolor{section in toc}{fg=uzh,bg=white}
\setbeamercolor{subsection in toc}{fg=uzh,bg=white}
\setbeamercolor{result}{fg=black, bg=yellow}



\addheadboxtemplate{\color[rgb]{1,1,1}}{\color{uzh} \underline{\hspace{10cm} \tiny Einführung in \LaTeX} }

\addfootboxtemplate{\color[rgb]{1,1,1}}{\color{black} %\tiny \quad  %bla 
\hfill \tiny
\insertframenumber / \inserttotalframenumber \hspace{5pt}}


%% set item distance
\newlength{\wideitemsep}
\setlength{\wideitemsep}{\itemsep}
\addtolength{\wideitemsep}{0.3em}
\let\olditem\item
\renewcommand{\item}{\setlength{\itemsep}{\wideitemsep}\olditem}

\newcommand{\ltx}{\LaTeX}

% \titlegraphic{\includegraphics[width=0.3\textwidth]{uzh_logo_e_pos}\hspace*{4.75cm}}

% \setlength{\parskip}{1.5ex}
\usepackage{amsmath}

\title[Einführung in \LaTeX] % (optional, only for long titles)
{Einführung in \LaTeX}
\author{Philipp Gloor}
\institute[] % (optional)
{philipp.gloor@gmail.com}
\date[Feburar 2016] % (optional)

\begin{document}

\begin{frame}
  \titlepage
\end{frame}

\begin{frame}[t]\frametitle{Kurs"ubersicht}
    
\begin{itemize}
  \item Werkzeuge (Distribution, Editor)
  \item Grundelemente f"ur den Textsatz mit \ltx
  \item Dokumentstruktur
  \item Titelseite
  \item Eigenheiten von \ltx (Anf"urungszeichen, Silbentrennung)
  \item Gliederungen
  \item Auflistungen
  \item Labels/Referenzen
  \item Seitenlayout/Geometrie
\end{itemize}
\end{frame}

\begin{frame}[t]\frametitle{Kurs"ubersicht}
    
\begin{itemize}
  \item Tabellen
  \item Figures
  \item Mathematischer Formelsatz
  \item Indexverzeichnis
  \item Bibliographie
  \item Tikz
  \item Beamer
  \item Github - Version Control
  \end{itemize}
\end{frame}

\begin{frame}{Was braucht es f"ur \LaTeX}
\LaTeX\  ist nicht ein herk"ommliches Programm wie z.B. Word von Microsoft. Das heisst, es gibt keine
\texttt{latex.exe} Datei, die man ausf"uhrt und dann mit \LaTeX\ arbeitet.
\begin{itemize}
  \item \LaTeX-Distribution (z.B. texlive oder Miktex)
  \item (optional) Editor
\end{itemize}

Die Distribution ist das Herzst"uck von \LaTeX. Ohne die Distribution geht nichts. Der Editor
ist lediglich ein Hilfsmittel, aber ein sehr wichtiges.
\end{frame}


\section{Distributionen}
\begin{frame}{Distributionen}
Wir werden die Distribution von texlive verwenden. Sie ist verf"ugbar f"ur alle g"angigen 
Betriebssysteme (Windows, Mac, Linux).

\begin{itemize}
  \item texlive f"ur Windows: \url{https://www.tug.org/texlive/acquire-netinstall.html}
  \item texlive f"ur Mac: \url{https://tug.org/mactex/mactex-download.html}
\end{itemize}
\end{frame}

\begin{frame}{Editor}
  Ein Editor ist nicht zwingend notwendig, erleichtert das Arbeiten mit \LaTeX aber immens. Es gibt eine vielzahl
  von verschiedenen Editoren (\url{https://en.wikipedia.org/wiki/Comparison_of_TeX_editors}).

  Eine Auswahl:
  \begin{itemize}
    \item Texmaker (\url{http://www.xm1math.net/texmaker/})
    \item TeXstudio - basiert auf texmaker (\url{http://texstudio.sourceforge.net/})
    \item Emacs
    \item Sublime Text
  \end{itemize}
\end{frame}

\begin{frame}{Workflow}
\LaTeX\ hat einen eigenen Workflow, der stark von dem von z.B. Word abweicht.

\begin{itemize}
  \item \texttt{.tex}: Quelldatei
  \pause\item Kompilieren (pdf Datei wird erstellt)
  \pause\item (automatisch) Hilfsdateien werden erstellt: z.B. \texttt{.toc, .aux, .lof}
  \pause\item Kompilieren (Hilfsdateien werden ins Dokument eingebunden)
\end{itemize}
  
\end{frame}

\begin{frame}{Aufbau eines \LaTeX\ Dokuments}
  Ein \ltx\ Dokument besteht grunds"atzlich aus zwei Teilen:
  \begin{itemize}
    \item Pr"aambel
    \item \textit{Document body}
  \end{itemize}

  Die Pr"aambel ist dort, wo alle Dokumentoptionen und Einstellungen gesetzt werden.
\end{frame}

\begin{frame}[fragile]\frametitle{Pr"aambel}

\begin{lstlisting}
\documentclass{article}
\usepackage[T1]{fontenc}
\usepackage[utf8]{inputenc}

\usepackage{amsmath} 
\usepackage{amsfonts}
\usepackage{amssymb}
\usepackage{fullpage}
\usepackage{graphicx}
\usepackage{tikz}
\end{lstlisting}

\end{frame}

\begin{frame}[fragile]\frametitle{Document body}
Neben der Pr"aambel gibt es wie erw"ahnt den \textit{Document body} Teil. Das ist der Teil zwischen den beiden
Befehlen \verb_\begin{document}_ und \verb_\end{document}_
\begin{lstlisting}
  \begin{document}
    Alles was hier steht, gehört zum Inhalt
  \end{document}
\end{lstlisting}

Jeglicher Inhalt der vor \verb_\begin{document}_ oder nach \verb_\end{document}_ kommt wird ignoriert.
\end{frame}

\begin{frame}[fragile]\frametitle{\ltx{} Befehle}
    
Wenn man in \ltx\ etwas fett setzen will, muss man folgenden Befehl verwenden:
\begin{lstlisting}
  \textbf{Dieser Text wird fett gesetzt}
\end{lstlisting} Und aussehen sollte es dann so:
\textbf{Dieser Text wird fett gesetzt}
\par
Dieser Befehl verwendet ein Argument, n"amlich den Teil in den geschwungenen Klammern.
\end{frame}


\begin{frame}[fragile]\frametitle{\ltx{} Befehle}
    
Manchmal habne Befehle auch optionale Argumente.

\begin{lstlisting}
  \usepackage[ngerman]{babel}

  \caption[Short cap.]{Long cap.}
\end{lstlisting}

\texttt{ngerman} definiert die Sprache, mit der das Paket \texttt{babel} arbeiten muss (das n steht f"ur die neue
Deutsche Rechtschreibung)
\end{frame}

\begin{frame}[fragile]\frametitle{Kommentare}
    
\ltx{} ist einer Programmiersprache sehr "ahnlich. Man k"onnte sagen, man programmiert seine arbeit.
Weil \ltx{} viele Befehle verwendet ist eine Kommtarfunktion sehr n"utzlich um gewisse Stellen zu annotieren.

\begin{lstlisting}
  \textbf{Fettgedruckter Text} % dieser Befehl setzt den Text in fetter Schrift
\end{lstlisting}

Das \% Zeichen wird verwendet um einen Kommentar zu erfassen. Alles was auf einer Zeile (automatischer
Zeilenumbruch inklusive) nach einem Prozentzeichen erscheint, wird ignoriert.

\end{frame}

\begin{frame}[fragile]\frametitle{Verbatim}
    
Wie kann ich einen \ltx{} Befehl im Klartext schreiben? Logischerweise, wenn ich einen \ltx-Befehl verwende
wird er von \ltx{} interpretiert. Wenn ich mir also eine Anleitung zu \ltx{} machen wollte, brauche ich einen
speziellen Befehl.

\begin{lstlisting}
  \verb_<beliebiger latex befehl>_
\end{lstlisting}

Der \verb_\verb_ Befehl hat anstelle von geschwungenen Klammern einen Underscore als `Klammer' (ein + als Klammer w"urde
auch funktionieren).

\end{frame}
\end{document}
