\documentclass[10pt]{beamer} %handout fuer eine gedruckte version der slides
\usepackage[T1]{fontenc}
\usepackage[utf8]{inputenc}
%\usetheme{CambridgeUS}
\usecolortheme{spruce}
\AtBeginSection[]
{
  \begin{frame}
    \frametitle{Table of Contents}
    \tableofcontents[currentsection]
  \end{frame}
}

\AtBeginSubsection[]
{
  \begin{frame}
    \frametitle{Table of Contents}
    \tableofcontents[currentsection,currentsubsection]
  \end{frame}
}


\title[Crisis] % (optional, only for long titles)
{The Economics of Financial Crisis}
\subtitle{Evidence from India}
\author[Author, Anders] % (optional, for multiple authors)
{F.~Author\inst{1} \and S.~Anders\inst{2}}
\institute[Universities Here and There] % (optional)
{
  \inst{1}%
  Institute of Computer Science\\
  University Here
  \and
  \inst{2}%
  Institute of Theoretical Philosophy\\
  University There
}
\date[KPT 2004] % (optional)
{Conference on Presentation Techniques, 2004}
\subject{Computer Science}

\begin{document}
\begin{frame}
\titlepage
\end{frame}


\section{Inhaltsverzeichnis}
\begin{frame}{Inhaltsverzeichnis}
Test
\end{frame}


\subsection{Literatur}
\begin{frame}{Literatur}
Literatur online: 
\begin{itemize}
\item \url{http://www.math.umbc.edu/~rouben/beamer/}
\end{itemize}
\end{frame}

\begin{frame}

\begin{itemize}
\item Beamer is a wonderful class
\pause \item One can make animations
\pause \item One uses the \textbf{pause} command, for example
\pause \item in order to bring in important ideas
\end{itemize}

\end{frame}
\end{document}
