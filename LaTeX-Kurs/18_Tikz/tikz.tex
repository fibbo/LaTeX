\documentclass[10pt, a4paper]{article}
\usepackage[T1]{fontenc}
\usepackage[utf8]{inputenc}
\usepackage[ngerman]{babel}
\usepackage{tikz}
\usepackage{url, hyperref}

\setlength{\parindent}{0pt}
\begin{document}

\section{Einfache Tikz Befehle}
Tikz steht fuer: \textit{Tikz ist kein Zeichenprogramm}, also ein rekursives Akronym. Tikz erstellt uns Vektorgrafiken. Vektorgrafiken lassen sich beliebig skalieren.

Eine ausf"uhrlichere Einleitung in Tikz findet ihr hier: \url{http://cremeronline.com/LaTeX/minimaltikz.pdf}


\subsection{Draw-Befehl}
\begin{tikzpicture}
\draw (0,0) --(1,2);
\end{tikzpicture}

\subsection{Optionen}

Man kann sich zum Beispiel ein Hilfsgitter zeichnen lassen:

\begin{tikzpicture}
\draw (0,0) --(1,2) -- (2,3) -- (1,0);
\draw[help lines, step=0.5] (0,0) grid (2,3);
\end{tikzpicture}

\begin{figure}[h]
\centering
\begin{tikzpicture}
\draw (0,0) -- (1,2) -- (2,3) -- (1,0);
\draw (0,3) -- (1.5,0.5);
\end{tikzpicture}
\caption{Tikz-Bilder kann man auch in einer Float verwenden}
\end{figure}

\subsection{Vektoren/Pfeile}

\begin{tikzpicture}
\draw [<->] (0,2) -- (0,0) -- (3,0);
\draw [->] (0,0) -- (2,2);
\draw [<-] (0,1) -- (2,3);
\end{tikzpicture}

\subsection{"Andern der Linienbreite}

\begin{tikzpicture}
\draw [ultra thick] (0,1) -- (2,1);
\draw [thick] (0,0.5) -- (2,0.5);
\draw [thin] (0,0) -- (2,0);
\end{tikzpicture}

M"ogliche Befehle um die Linienbreite zu "andern:

\begin{itemize}
\item ultra thin
\item very thin
\item thin
\item semithick
\item thick
\item very thick
\item ultra thick
\end{itemize}

\section{Funktionen zeichnen mit Tikz}
\begin{center}
\begin{tikzpicture}[xscale=13,yscale=3.8] % Man kann die Zeichnung auch skalieren
\draw [<->] (0,0.8) -- (0,0) -- (0.5,0);
\draw[green, ultra thick, domain=0:0.5] plot (\x, {0.025+\x+\x*\x});
\end{tikzpicture}
\end{center}

\begin{figure}[h]
\centering
\begin{tikzpicture}[yscale=1.5] % Hier ist nur die y-Achse skaliert
\draw [help lines, <->] (0,0) -- (6.5,0);
\draw [help lines, ->] (0,-1.1) -- (0,1.1);
\draw [green,domain=0:2*pi] plot (\x, {(sin(\x r)* ln(\x+1))/2});
\draw [red,domain=0:pi] plot (\x, {sin(\x r)});
\draw [blue, domain=pi:2*pi] plot (\x, {cos(\x r)*exp(\x/exp(2*pi))});
\end{tikzpicture}
\caption{In Tikz kann man Winkel in Grad oder Radiant angeben. Will man es in Radiant angeben, muss man dies mit einen \texttt{r} so kennzeichnen.}
\end{figure}

\section{Verwendung von 'Nodes'}
\begin{tikzpicture}
\draw [thick, <->] (0,2) -- (0,0) -- (2,0);
\node at (1,1) {yes};
\end{tikzpicture}

\vspace{5mm}

\begin{tikzpicture}
\draw [thick, <->] (0,2) -- (0,0) -- (2,0);
\draw[fill] (1,1) circle [radius=0.025];
\node [below] at (1,1) {below};
\node [above] at (1,1) {above};
\node [left] at (1,1) {left};
\node [right] at (1,1) {right};
\end{tikzpicture}

\section{Sich wiederholende Strukturen in Tikz}
\begin{tikzpicture}[yscale = 0.2]
\foreach \x in {1,2,...,6}{
\draw (\x,-0.5) -- (\x,1);
\draw (0, \x) -- (0.5,\x);
}
\end{tikzpicture}

\newpage

\begin{tikzpicture}[cap=round]
% Colors
\colorlet{anglecolor}{green!50!black}
\colorlet{bordercolor}{black}

%Configuration: change this to define number of intersections: 
% 5 degree mean 360/10 = 36 elements
\def\alpha{2} % degree
\def\layer{5}

\begin{scope}[scale=5]
% Radius R = 1

% The figure is constructed by intersecting circles Cx of radius R.
%  M_Cx lies on the circle C with a radius \alpha degree from the outer circle R 
%  and a distance defined by \alpha degree.

% It is sufficent to calculate one special M_C, which is intersecting the x-axis 
% at distance R from (0,0).
\pgfmathsetmacro\sinTriDiff{sin(60-\alpha)}
\pgfmathsetmacro\cosTriDiff{1-cos(60-\alpha)}
% The distance from the (0,0).
\pgfmathsetmacro\radiusC{sqrt(\cosTriDiff*\cosTriDiff + \sinTriDiff*\sinTriDiff)}
% Angle of M_C (from x-axis)
\pgfmathsetmacro\startAng{\alpha + atan(\sinTriDiff/\cosTriDiff)}

% The segment layer are \alpha degree apart
\pgfmathsetmacro\al{\alpha*\layer}

% For each segment create the intersection parts of the circles by using arcs
\foreach \x in {0,\alpha,...,\al}
{
 % Calculate the polar coordiantes of M_Cx. We take the M_C from above 
 % and  can calculate all other M_Cx by adding \alpha
 \pgfmathsetmacro\ang{\x + \startAng}
 % From ths we get the (x,y) coordinates
 \pgfmathsetmacro\xRs{\radiusC*cos(\ang)}
 \pgfmathsetmacro\yRs{\radiusC*sin(\ang)}

 % Now we intersect each new M_C with the x-axis:
 % We can find the radius of concentric inner circles
 \pgfmathsetmacro\radiusLayer{\xRs + sqrt( 1 - \yRs*\yRs )}

 % To calculate angles for the arcs later, this angle is needed
 \pgfmathsetmacro\angRs{acos(\yRs)}

% We need to have the angle from the previous loop as well
 \pgfmathsetmacro\angRss{acos(\radiusC*sin(\ang-\alpha))}


 % Add some fading by \ang
 \colorlet{anglecolor}{black!\ang!green}

 % The loop needs to run a whole.  
 % We don't want to cope with angles > 360 degree, adapt the limits. 
 \pgfmathsetmacro\step{2*\alpha - 180}
 \pgfmathsetmacro\stop{180-2*\alpha}
 \foreach \y in {-180, \step ,..., \stop}
 {
   \pgfmathsetmacro\deltaAng{\y-\x}
   % This are the arcs which are definied by the intersection of 3 circles 
   \filldraw[color=anglecolor,draw=bordercolor] 
       (\y-\x:\radiusLayer)    
               arc (-90+\angRs+\deltaAng : \alpha-90+\angRss+\deltaAng :1) 
               arc (\alpha+90-\angRss+\deltaAng : 2*\alpha+90-\angRs+\deltaAng :1)
               arc (\deltaAng+2*\alpha : \deltaAng : \radiusLayer);
 }


 % helper circles  & lines
 %\draw[color=gray] (\xRs,\yRs) circle (1);
 %\draw[color=gray] (\xRs,-\yRs) circle (1);
 %\draw[color=blue] (0,0) circle (\radiusLayer);
 %\draw[color=blue, very thick] (0,0) -- (0:1);
 %\draw[color=blue, very thick] (0,0) -- (\ang:\radiusC) -- (\xRs,0);  
 %\draw[color=blue, very thick] (\xRs,\yRs) -- (0:\radiusLayer);
 %\filldraw[color=blue!20, very thick] (\xRs,\yRs) -- 
 % (\xRs,\yRs-0.3) arc (-90:-90+\angRs:0.2) -- cycle;

}
% Additional inner decoration element
\pgfmathsetmacro\xRs{\radiusC*cos(\al+\startAng)}
\pgfmathsetmacro\yRs{\radiusC*sin(\al+\startAng)}
\pgfmathsetmacro\radiusLayer{\xRs + sqrt( 1 - \yRs*\yRs )}
\draw[line width=2, color=bordercolor] (0,0) circle (.8*\radiusLayer);
\pgfmathsetmacro\radiusSmall{.7*\radiusLayer}
% There are six elements to create. Avoid angles >360 degree.
\foreach \x in {-60,0,...,240}
{
   \fill[color=anglecolor] (\x:\radiusSmall) arc (-180+\x+60: -180+\x: \radiusSmall)
                            arc (0+\x: -60+\x: \radiusSmall)
                            arc (120+\x: 60+\x: \radiusSmall); 
}
% The outer decoration
\foreach \x in {0, 4, ..., 360}
{
 \fill[color=anglecolor] (\x:1) -- (\x+3:1.05) -- (\x+5:1.05) -- (\x+2:1) -- cycle;
 \fill[color=anglecolor] (\x+5:1.05) -- (\x+7:1.05) -- (\x+4:1.1) -- (\x+2:1.1) -- cycle;
}
\draw[line width=1, color=bordercolor] (0,0) circle (1);
\draw[line width=1, color=bordercolor] (0,0) circle (1.1);
\end{scope}

\end{tikzpicture}
\end{document}
